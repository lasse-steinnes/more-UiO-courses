\documentclass[11pt,norsk]{article}
\usepackage{mathptmx}
\usepackage[T1]{fontenc}
\usepackage{babel}
\usepackage{amsmath}
\usepackage{amsfonts}
\usepackage{amsthm}
\usepackage[colorlinks]{hyperref}
\usepackage{palatino}
\usepackage[document]{ragged2e}
\usepackage{float}

% Standard stuff
\usepackage{amsmath,graphicx,varioref,verbatim,amsfonts,geometry}
% colors in text
\usepackage[usenames,dvipsnames,svgnames,table]{xcolor}
% Hyper refs
\usepackage[colorlinks]{hyperref}
\usepackage[document]{ragged2e}

% Document formatting
\setlength{\parindent}{0mm}
\setlength{\parskip}{1.5mm}

%Color scheme for listings
\usepackage{textcomp}
\definecolor{listinggray}{gray}{0.9}
\definecolor{lbcolor}{rgb}{0.9,0.9,0.9}

%lengden av en vektor
\newcommand{\norm}[1]{\lvert #1 \rvert}

%Listings configuration
\usepackage{listings}
%Hvis du bruker noe annet enn python, endre det her for å få riktig highlighting.
\lstset{
	backgroundcolor=\color{lbcolor},
	tabsize=4,
	rulecolor=,
	language=python,
        basicstyle=\scriptsize,
        upquote=true,
        aboveskip={1.5\baselineskip},
        columns=fixed,
	numbers=left,
        showstringspaces=false,
        extendedchars=true,
        breaklines=true,
        prebreak = \raisebox{0ex}[0ex][0ex]{\ensuremath{\hookleftarrow}},
        frame=single,
        showtabs=false,
        showspaces=false,
        showstringspaces=false,
        identifierstyle=\ttfamily,
        keywordstyle=\color[rgb]{0,0,1},
        commentstyle=\color[rgb]{0.133,0.545,0.133},
        stringstyle=\color[rgb]{0.627,0.126,0.941}
        }
        
\newcounter{subproject}
\renewcommand{\thesubproject}{\alph{subproject}}
\newenvironment{subproj}{
\begin{description}
\item[\refstepcounter{subproject}(\thesubproject)]
}{\end{description}}

%Lettering instead of numbering in different layers
%\renewcommand{\labelenumi}{\alph{enumi}}
%\renewcommand{\thesubsection}{\alph{subsection}}


%opening
\begin{document}

\title{Oblig 2\\
\large{MAT1110}
\author{Lasse Steinnes}
\date{11. April 2018} 
\maketitle

\section*{Oppgave 1}
\subsection*{a)}
Dersom $f: \textbf{R}^2 \rightarrow \textbf{R}$ er gitt ved 
$$
f(x,y) = \frac{1}{x^4 + 2x^{2}y^{2} + y^{4} + 1},
$$

s� blir det uegentlige dobbeltintegralet over $\textbf{R}^2$ for $f(x,y)$ i polarkoordinater: 

$$
\int_{0}^{\infty}\int_{0}^{2\pi}  \frac{r}{r^4\cos^4{\theta} + 2r^4\cos{\theta}\sin^2{\theta} + r^{4}\sin^{4}{\theta} +1 } dr d\theta,
$$
der jacobideterminanten for polarkoordinater, r, er multiplisert inn i telleren.

Omskriving gir 
$$
\int_{0}^{\infty}\int_{0}^{2\pi}  \frac{r}{r^4(\cos^4\theta + 2\cos{\theta}\sin^2{\theta} + \sin^{4}{\theta}) +1} dr d\theta, 
$$
og siden parentesen kan skrives som et kvadrat, s� er integralet
$$
 		= \int_{0}^{\infty}\int_{0}^{2\pi}  \frac{r}{r^4(\cos^2\theta + sin^2{\theta})^2 +1} dr d\theta.
$$
Fordi identiteten $cos^2\theta + sin^2{\theta} = 1$, blir integralet
$$
\int_{0}^{\infty}\int_{0}^{2\pi} \frac{r}{r^4 +1} dr d\theta.
$$

Gj�res substitusjonen $u = r^2$, s� er $du = 2r dr$, som betyr at
$$
\frac{1}{2} \int_{0}^{\infty}\int_{0}^{2\pi} \frac{1}{u^2 +1} du d\theta,
$$

Integranden er den deriverte til $\arctan(u)$, slik at 
$$
\frac{1}{2} \int_{0}^{2\pi} \arctan(u) \right |_{0}^{\infty} d\theta = \frac{1}{2} \int_{0}^{2\pi} \lim_{u \rightarrow \infty} \arctan(u) - \arctan(0)d\theta,
$$
og fordi $\arctan(0) = 0$, og $\lim_{u \rightarrow \infty} \arctan(u) = \frac{\pi}{2}$, blir integralet,

$$
\frac{1}{2} \int_{0}^{2\pi}  \frac{\pi}{2} d\theta =  \frac{1}{2}*\frac{\pi}{2}*2\pi = \frac{\pi^2}{2}
$$

\subsection*{b)}
Dersom x,y $\epsilon \left [-k,k  \right ]$, og oppdelingen av disse intervallene deles opp med lik steglengde, blir partisjonen $\prod$: 
$$
-k = x_{0} < x_{1} < ... < x_{n-1} < x_{n} = k,
$$
$$
-k = y_{0} < y_{1} < ... < y_{n-1} < y_{n} = k.
$$

St�rrelsen p� intervallet er fra -k til k er 2k, slik at dersom punkter $c_{ij}$  i et  utplukk U velges til endepunktet i arealelementet $R_{ij}$, blir $c_{ij} = (- k + \frac{2ki}{n}, -k + \frac{2kj}{n})$, der i,j $\epsilon [1,n  \right ]$.  Riemannsummen er definert som

$$
R(\Pi,U) = \sum_i^n\sum_j^m f(c_{ij}) |\left R_{ij} \right |
$$
slik at 
$$
R(\Pi,U) = R_n^k =  \sum_i^n\sum_j^n f(- k + \frac{2ki}{n}, -k + \frac{2kj}{n})(\frac{2k}{n})^{2}
$$
Etterhvert som partisjonen blir finere og finere, dvs. $\lim _{n \rightarrow \infty}$, konvergerer riemannsummen mot integralet oppgitt i a), jamf�r �vre og nedre trappesum.

\subsection*{c)}
Gitt f definert som en funksjon, og n og k som heltall, kan man beregne riemannsummen numerisk i python ved hjelp av f�lgende algoritme:

\begin{centre}
$$
\text{sum f = 0}
$$
$$
    \text{for i in range(1,n+1):}
$$
$$
        \text{for j in range(1,n+1):}
$$
$$
           \text{dx = - k + 2*k*i/n}
$$
$$
            \text{dy = - k + 2*k*j/n}
$$
$$
            \text{sum f = f(dx,dy)*(2*k/n)**2}
$$
\end{centre}

Algoritmen tar summen over alle i,j-er og evaluerer funksjonen for alle $c_{i,j}$ og samler summen i objektet "sum f". For hele koden og kj�reeksempel, se vedlegg 1. 

\subsection*{d)}
Verdien for n og k ble bestemt slik at riemannsummen $R_n^k$ tiln�rmer seg eksaktverdien med feil mindre enn 1/10. Dette ble utf�rt ved hjelp av en if-setning med absoluttverdien av differansen mellom eksakt og tiln�rmet verdi (vedlegg 2). $n = 22$ og $k = 19$ vil gi en $\varepsilon < \frac{1}{10}$.

\section*{Oppgave 2}
Gitt at planeten Solaris har en fast kjerne med radius R, og dekket med et hav med dybde D, i tillegg til en massetetthet 
$$
\rho (h) = \frac{\alpha}{R + h},
$$
er massen gitt som volumintegralet 
$$
\int\int\int_{V} \rho (h) dV.
$$

Skrives integranden om til kulekoordinater gir dette 
$$
\int_0^\pi \int_D^R \int_0^{2\pi} \frac{\alpha (h^{2}\sin\phi)}{R +h} d\phi dh d\theta,
$$
som f�lge av at jacobideterminanten til kulekoordinater er $h^{2}\sin\phi$. Fra analysens fundamentalteorem har vi at 
$$
- \int_D^R \int_0^{2\pi} \left [ \frac{\alpha (h^{2}\cos\phi)}{R +h} \right ]_{0}^{\pi} dh d\theta.
$$ 

Siden $\cos(\pi) = -1$ og $\cos(0) = 1$ reduseres integralet til 
$$
2\alpha \int_D^R \int_0^{2\pi} \frac{h^2}{R + h} dh d\theta.
$$

Fordi graden i telleren er mindre enn i nevneren er det lurt � benytte polynomdivisjon til � redusere graden i br�ken. Polynomdivisjon ble foretatt to ganger, og integranden ble redusert til
$$
h^2 : (R + h) = h - \frac{Rh}{R +h} = h - R + \frac{R^2}{R + h}
$$

Da blir integralet 
$$
2\alpha \int_D^R \int_0^{2\pi} h - R + \frac{R^2}{R + h} dh d\theta,
$$
og fordi R kun er en konstant blir dette
$$
 \int_0^{2\pi} \left [\frac{h}{2} - Rh + {R^2}ln{R + h} \right ]_{R}^{R+D} d\theta.
$$
Skrives dette ut kan ledd strykes mot hverandre, og benyttes regneregler for logaritmefunksjonen blir integralet som st�r igjen
$$d
 2\alpha  \int_0^{2\pi} \frac{D^2}{2} + R^{2}ln(\frac{2R + D}{2R}) d\theta.
$$

$$
 			= 2\alpha 2\pi (\frac{D^2}{2} + R^{2}ln(\frac{2R + D}{2R})) - 0 
			= 4 \alpha \pi (\frac{D^2}{2} + R^{2}ln(\frac{2R + D}{2R}))
$$ 

\section*{Oppgave 3}
Dersom M er kvadratisk, og $\overrightarrow{x}$ og $\overrightarrow{y}$ er line�rt uavhengige egenvektorer med henholdsvis egenverdier $\lambda$ og $\mu$ for M, kan en vilk�rlig $\overrightarrow{v}$ skrives som en line�rkombinasjon 
$$
\overrightarrow{v} = a\overrightarrow{x} + b\overrightarrow{y}. 
$$
Dersom $\overrightarrow{v}$ ogs� er en egenvektor med egenverdi $\phi$, m� vi ha at n�r M multipliseres inn fra venstre 
$$
M\overrightarrow{v} = M(a\overrightarrow{x} + b\overrightarrow{y}). 
$$
Fra distributiv lov, s� 
$$
M\overrightarrow{v} = Ma\overrightarrow{x} + Mb \overrightarrow{y}. 
$$
Satt inn for egenverdiene har vi at 
$$
\phi \overrightarrow{v} = \lambda a\overrightarrow{x} + \mu b\overrightarrow{y}
$$
Flytter vi over venstresida 
$$
\overrightarrow{0} = \lambda a\overrightarrow{x} + \mu b\overrightarrow{y} - \phi \overrightarrow{v}.
$$
Satt inn for $\overrightarrow{v}$ gir dette 
$$
\overrightarrow{0} = \lambda a\overrightarrow{x} + \mu b\overrightarrow{y} - \phi(a\overrightarrow{x} + b\overrightarrow{y}).
$$
Trekkes sammen leddene f�r vi at 
$$
\overrightarrow{0} = (\lambda - \phi)a\overrightarrow{x} + (\mu - \phi)b\overrightarrow{y}.
$$
Siden $\lambda \neq \mu$, kan ikke $\lambda - \phi$ og $\mu - \phi$ v�re null samtidig, og derfor m� enten $a = 0$ eller $b = 0$, som var det som skulle bevises. Dette f�lger ogs� fra setning 4.6.6, om man l�ser for $(\lambda - \phi)a = 0$ og $(\mu - \phi)b = 0$.

\section*{Vedlegg}
\subsection*{1) Numerisk beregning av integralet med riemannssum}
\lstinputlisting[language=Python, firstline=1, lastline=25]{oblig2mat1110.py}
\subsection*{2) Numerisk beregning med $\varepsilon < \frac{1}{10}$}
\lstinputlisting[language=Python, firstline=26, lastline=48]{oblig2mat1110.py}

\end{document}





